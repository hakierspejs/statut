\pdfinfo{/CreationDate(D:19900101000000Z00'00')/ModDate(D:19900101000000Z00'00')}
\pdftrailerid{}
\documentclass[chapterprefix,notitlepage]{article}
\usepackage[utf8]{inputenc}
\usepackage[polish]{babel}
\usepackage{polski}
\usepackage[a4paper,margin=1in]{geometry}
\usepackage[shortlabels]{enumitem}
\usepackage{mdwlist}


\renewcommand{\labelenumi}{\textbf{§ \arabic{enumi}.}}
\renewcommand{\labelenumii}{\arabic{enumii}.}
\renewcommand{\labelenumiii}{\arabic{enumiii})}
\renewcommand{\labelenumiv}{\alph{enumiv})}

\renewcommand{\thesection}{Rozdział \Roman{section}}


\begin{document}

\title{Statut Stowarzyszenia Hakierspejs Łódź}
\author{}
\date{}
\maketitle


\section{Postanowienia ogólne}

\begin{enumerate}

	\item Stowarzyszenie, w dalszych postanowieniach zwane Stowarzyszeniem, nosi nazwę „Stowarzyszenie Hakierspejs Łódź”.
	
	\item \begin{enumerate}
		\item Terenem działania Stowarzyszenia jest Rzeczpospolita Polska, a jego siedzibą jest miasto Łódź.
		\item Dla realizacji celów statutowych Stowarzyszenie może działać na terenie innych państw z~poszanowaniem tamtejszego prawa.
	\end{enumerate}
	
	\item Stowarzyszenie jest zawiązane na czas nieograniczony. Posiada osobowość prawną. Działa na podstawie przepisów ustawy z dnia 7 kwietnia 1989 r. Prawo o stowarzyszeniach (Dz.U. z 2001, Nr 79, poz. 855 z późn. zm.) oraz niniejszego statutu.
	
	\item Stowarzyszenie może być członkiem innych krajowych i międzynarodowych organizacji o podobnych celach.
	
	\item \begin{enumerate}
		\item Działalność Stowarzyszenia oparta jest przede wszystkim na pracy społecznej członków.
		\item Członkami Stowarzyszenia mogą być obywatele polscy oraz cudzoziemcy, włącznie z osobami nie mającymi miejsca zamieszkania na terytorium Rzeczypospolitej Polskiej.
	\end{enumerate}
	
	\item Do prowadzenia swoich spraw Stowarzyszenie może zatrudniać pracowników.
	
	
\suspend{enumerate}
\section{Cele i środki działania}
\resume{enumerate}

	\item \begin{enumerate}
		\item Celem stowarzyszenia jest:
		\begin{enumerate}
			\item działalność edukacyjna, naukowa i oświatowo-wychowawcza, wspieranie idei nauki otwartej wobec obywateli i przez obywateli tworzonej,
			\item popularyzacja nauki i metody naukowej,
			\item wspieranie postaw obywatelskich, w szczególności wśród młodzieży,
			\item upowszechnianie i ochrona praw człowieka oraz swobód obywatelskich,
			\item upowszechnianie i ochrona wolności osobistej i gospodarczej,
			\item promowanie wolności w zakresach: światopoglądu, wolności do wyboru alternatywnej edukacji, prowadzenia działalności gospodarczej, wolności informacji, wolnego i otwartego oprogramowania oraz otwartych technologii (ang. open hardware),
			\item podejmowanie działań na rzecz pomocy, rozwoju i aktywizacji oraz wspierania postaw przedsiębiorczych, w szczególności wśród młodzieży, osób niepełnosprawnych i wykluczonych zawodowo,
			\item wspieranie aktywności społecznej i kulturalnej, w szczególności wśród młodzieży,
			\item upowszechnianie kultury fizycznej, sportu i turystyki.
		\end{enumerate}
		\item Stowarzyszenie swe cele realizuje poprzez:
		\begin{enumerate}
			\item tworzenie i utrzymywanie infrastruktury stymulującej rozwój projektów, organizującej i~użyczającej potrzebnych narzędzi,
			\item prowadzenie spotkań, konferencji, seminariów, wykładów i szkoleń,
			\item organizowanie konkursów i imprez promocyjnych oraz działalność kulturalną, w szczególności organizowanie wystaw, prezentacji, projekcji oraz happeningów,
			\item udział w imprezach promujących naukę,
			\item realizację i wspieranie projektów naukowych,
			\item działalność edukacyjną,
			\item działalność publicystyczną i wydawniczą,
			\item integrację środowiska akademickiego, naukowego i przemysłowego,
			\item współpracę z krajowymi i zagranicznymi organizacjami.
		\end{enumerate}
	\end{enumerate}
	
	
\suspend{enumerate}
\section{Członkowie -- prawa i obowiązki}	
\resume{enumerate}
	
	\item Członkami Stowarzyszenia mogą być osoby fizyczne oraz prawne.
	
	\item Stowarzyszenie posiada członków:
	\begin{enumerate}[1)]
		\item zwyczajnych,
		\item wspierających.
	\end{enumerate}

	\item \begin{enumerate}
		\item Członkiem zwyczajnym Stowarzyszenia może być każda osoba fizyczna, która spełni wszystkie poniższe warunki:
		\begin{enumerate}
			\item złoży deklarację członkowską na piśmie,
			\item przedstawi pozytywną opinię co najmniej jednego członka zwyczajnego Stowarzyszenia,
			\item uiści składkę członkowską.
		\end{enumerate}
		\item Osoba fizyczna staje się członkiem zwyczajnym Stowarzyszenia po zaakceptowaniu pisemnej deklaracji członkowskiej uchwałą Zarządu Stowarzyszenia.
		\item Członkowie zwyczajni mają prawo do:
		\begin{enumerate}
			\item korzystania z dorobku, majątku i wszelkich form działalności Stowarzyszenia,
			\item udziału w zebraniach, wykładach oraz imprezach organizowanych przez Stowarzyszenie,
			\item zgłaszania wniosków dotyczących działalności Stowarzyszenia,
			\item biernego i czynnego uczestniczenia w wyborach do władz Stowarzyszenia.
		\end{enumerate}
		\item Członkowie zwyczajni mają obowiązek:
		\begin{enumerate}
			\item przestrzegania statutu i uchwał władz Stowarzyszenia,
			\item regularnego opłacania składek.
		\end{enumerate}
		\item Zarząd może w uzasadnionych przypadkach czasowo zwolnić określonego członka z konieczności płacenia składek.
	\end{enumerate}

	\item \begin{enumerate}
		\item Członkiem wspierającym Stowarzyszenia może być każda osoba fizyczna lub prawna po złożeniu pisemnej deklaracji członkowskiej i zaakceptowaniu jej uchwałą Zarządu Stowarzyszenia.
		\item Członek wspierający Stowarzyszenia będący osobą prawną może zadeklarować pomoc finansową, rzeczową lub merytoryczną w realizacji celów Stowarzyszenia w formie i wielkości uzgodnionej z Zarządem Stowarzyszenia, potwierdzonej umową.
		\item Członkowie wspierający nie posiadają biernego oraz czynnego prawa wyborczego, mogą jednak dobrowolnie brać udział z głosem doradczym w statutowych władzach Stowarzyszenia. Poza tym posiadają takie prawa, jak członkowie zwyczajni.
		\item Członek wspierający ma obowiązek wywiązywania się z zadeklarowanych świadczeń oraz przestrzegania statutu i uchwał władz Stowarzyszenia.
	\end{enumerate}
	
		\item \begin{enumerate}
			\item Utrata członkostwa następuje na skutek:
			\begin{enumerate}
				\item pisemnej rezygnacji złożonej na ręce Zarządu,
				\item wykluczenia przez Zarząd z powodu:
				\begin{enumerate}
					\item łamania statutu lub nieprzestrzegania uchwał władz Stowarzyszenia,
					\item unikania lub notorycznego braku udziału w pracach Stowarzyszenia,
					\item braku wpłat składek członkowskich za okres trzech miesięcy,
					\item łamania zasad współżycia społecznego na szkodę Stowarzyszenia,
				\end{enumerate}
				\item utraty praw publicznych na mocy prawomocnego wyroku sądu,
				\item śmierci członka lub utraty osobowości prawnej przez osobę prawną.
			\end{enumerate}
			\item Od uchwały Zarządu w sprawie pozbawienia członkostwa zainteresowanemu przysługuje odwołanie do Walnego Zebrania Członków. Odwołanie powinno zostać przekazane Zarządowi w formie pisemnej w terminie 14 dni od chwili poinformowania zainteresowanego o treści uchwały Zarządu. Uchwała Walnego Zebrania Członków jest ostateczna i wchodzi w życie w~trybie natychmiastowym.
		\end{enumerate}


\suspend{enumerate}
\section{Władze Stowarzyszenia}
\resume{enumerate}

	\item Władzami Stowarzyszenia są:
	\begin{enumerate}[1)]
		\item Walne Zebranie Członków,
		\item Zarząd,
		\item Komisja Rewizyjna.
	\end{enumerate}
	
	\item Kadencja Zarządu i Komisji Rewizyjnej Stowarzyszenia trwa dwa lata.
	
	\item Uchwały wszystkich władz Stowarzyszenia zapadają zwykłą większością głosów przy obecności co~najmniej połowy członków uprawnionych do głosowania, stanowiących kworum, chyba że dalsze postanowienia statutu stanowią inaczej.
	
	\item \begin{enumerate}
		\item Walne Zebranie Członków jest najwyższą władzą Stowarzyszenia. W Walnym Zebraniu Członków biorą udział:
		\begin{enumerate}
			\item z głosem stanowiącym – członkowie zwyczajni,
			\item z głosem doradczym – członkowie wspierający oraz zaproszeni goście.
		\end{enumerate}
		\item Walne Zebranie Członków jest zwoływane w trybie zwyczajnym lub nadzwyczajnym.
		\item Zebranie odbywa się w siedzibie Stowarzyszenia lub jest przeprowadzane z wykorzystaniem środków komunikacji elektronicznej.
		\item Zarząd powiadamia członków Stowarzyszenia za pośrednictwem poczty elektronicznej, w terminie uzależnionym od trybu organizacji zebrania, o terminie zebrania, planowanym porządku obrad oraz:
				\begin{enumerate}
					\item jego szczegółowej lokalizacji – w przypadku zebrania przeprowadzanego w siedzibie Stowarzyszenia,
					\item sposobie dołączenia do spotkania – w przypadku zebrania przeprowadzanego z wykorzystaniem środków komunikacji elektronicznej.
				\end{enumerate}
	\end{enumerate}
	
	\item \begin{enumerate}
		\item Walne Zebranie Członków w trybie zwyczajnym zwoływane jest przez Zarząd Stowarzyszenia nie rzadziej, niż raz na dwa lata.
		\item Zarząd powiadamia członków Stowarzyszenia o zebraniu z wyprzedzeniem co najmniej 14 dni kalendarzowych.
		\item Podczas obrad Walnego Zebrania Członków zwołanego w trybie zwyczajnym nie jest wymagane kworum.
	\end{enumerate}
	
	\item \begin{enumerate}
		\item Walne Zebranie Członków w trybie nadzwyczajnym zwoływane jest przez Zarząd Stowarzyszenia w dowolnym terminie z inicjatywy własnej lub na wniosek:
		\begin{enumerate}
			\item Komisji Rewizyjnej lub
			\item co najmniej 1/3 ogólnej liczby członków zwyczajnych Stowarzyszenia lub
			\item co najmniej dziesięciu członków zwyczajnych Stowarzyszenia.
		\end{enumerate}
		\item Wniosek o zwołanie Walnego Zebrania Członków w trybie nadzwyczajnym powinien zawierać propozycje terminów oraz porządku obrad.
		\item Dla zwołania Walnego Zebrania Członków w trybie nadzwyczajnym podaje się dwa terminy odległe o co najmniej 2 dni robocze, ale nie więcej, niż 14 dni kalendarzowych. Zarząd powiadamia członków Stowarzyszenia o zebraniu z wyprzedzeniem co najmniej 2 dni roboczych przed jego pierwszym terminem, przypadającym na nie później, niż 30 dni kalendarzowych od~daty wpływu wniosku do Zarządu.
		\item Aby obrady Walnego Zebrania Członków zwołanego w trybie nadzwyczajnym mogły rozpocząć się w pierwszym terminie, wymagana jest obecność kworum. W razie braku kworum, zebranie odbywa się w drugim terminie. Wówczas nie jest wymagane kworum. Walne Zebranie zwołane w trybie nadzwyczajnym obraduje wyłącznie nad sprawami, dla których zostało zwołane.
	\end{enumerate}
	
	\item \begin{enumerate}
		\item Uchwały Walnego Zebrania Członków zapadają zwykłą większością głosów w trybie głosowania jawnego lub, na żądanie dowolnego uczestnika, tajnego.
		\item Uchwały o wyborze i odwoływaniu władz Stowarzyszenia oraz o zmianach statutu wymagają bezwzględnej większości głosów.
	\end{enumerate}
	
	\item Do kompetencji Walnego Zebrania Członków należą:
	\begin{enumerate}[1)]
	\item określanie głównych kierunków działania i rozwoju Stowarzyszenia,
	\item uchwalanie zmian statutu,
	\item wybór i odwoływanie Zarządu oraz Komisji Rewizyjnej,
	\item udzielanie Zarządowi absolutorium na wniosek Komisji Rewizyjnej,
	\item ustalanie wysokości składek członkowskich,
	\item rozpatrywanie i zatwierdzanie sprawozdań władz Stowarzyszenia,
	\item rozpatrywanie wniosków i postulatów zgłoszonych przez członków Stowarzyszenia lub jego władze,
	\item rozpatrywanie odwołań od uchwał Zarządu,
	\item podjęcie uchwały o rozwiązaniu Stowarzyszenia i przeznaczeniu jego majątku,
	\item podejmowanie uchwał w każdej sprawie wniesionej pod obrady, we wszystkich sprawach nie~zastrzeżonych do kompetencji innych władz Stowarzyszenia.
	\end{enumerate}
	
	\item \begin{enumerate}
		\item Zarząd jest powołany do kierowania całą działalnością Stowarzyszenia zgodnie z uchwałami Walnego Zebrania Członków, a także reprezentuje Stowarzyszenie wobec organów administracji i sądów.
		\item Zarząd składa się z od trzech do siedmiu osób.
		\item Posiedzenia Zarządu odbywają się w miarę potrzeb, nie rzadziej jednak, niż raz na dwa lata. Posiedzenia Zarządu zwołuje dwóch członków zarządu działających łącznie.
		\item Do kompetencji Zarządu należą:
		\begin{enumerate}
			\item realizacja celów Stowarzyszenia,
			\item wykonywanie uchwał Walnego Zebrania Członków,
			\item sporządzanie planów pracy i budżetu,
			\item sprawowanie zarządu nad majątkiem Stowarzyszenia,
			\item podejmowanie uchwał o zarządzaniu majątkiem Stowarzyszenia,
			\item reprezentowanie Stowarzyszenia wobec organów administracji i sądów,
			\item zwoływanie Walnego Zebrania Członków,
			\item przyjmowanie i skreślanie członków,
			\item składanie sprawozdań ze swojej działalności na Walnym Zebraniu Członków,
			\item sporządzanie rocznego sprawozdania finansowego.
		\end{enumerate}
	\end{enumerate}

	\item \begin{enumerate}
		\item Komisja Rewizyjna powoływana jest do sprawowania kontroli nad działalnością Stowarzyszenia.
		\item Komisja Rewizyjna składa się z od trzech do pięciu osób.
		\item Do kompetencji Komisji Rewizyjnej należy:
		\begin{enumerate}
			\item kontrolowanie działalności Zarządu,
			\item składanie wniosków z kontroli na Walnym Zebraniu Członków,
			\item prawo do wystąpienia z wnioskiem o zwołanie Walnego Zebrania Członków oraz zebrania Zarządu,
			\item składanie wniosków o absolutorium dla Zarządu Stowarzyszenia,
			\item składanie sprawozdań ze swojej działalności na Walnym Zebraniu Członków,
			\item zatwierdzanie rocznego sprawozdania finansowego.
		\end{enumerate}
	\end{enumerate}

	\item W razie, gdy skład Zarządu lub Komisji Rewizyjnej ulegnie zmniejszeniu w czasie trwania kadencji, uzupełnienie ich składu może nastąpić w drodze kooptacji spośród członków zwyczajnych Stowarzyszenia, której dokonują pozostali członkowie organu, który uległ zmniejszeniu. W tym trybie można powołać nie więcej, niż połowę składu organu.


\suspend{enumerate}
\section{Majątek i fundusze}
\resume{enumerate}

	\item Majątek Stowarzyszenia powstaje ze:
	\begin{enumerate}[1)]
		\item składek członkowskich,
		\item subwencji, darowizn, spadków i zapisów,
		\item wpływów z odpłatnej działalności statutowej,
		\item wpływów z ofiarności publicznej,
		\item dochodów z majątku, odsetek oraz kapitału,
		\item dotacji i kontraktów państwowych,
		\item wpływów z loterii, aukcji i sponsoringu,
		\item zbiórek publicznych.
	\end{enumerate}
	
	\item \begin{enumerate}
		\item Organem kompetentnym w zakresie zarządzania majątkiem Stowarzyszenia jest Zarząd.
		\item Zarząd zobowiązany jest dołożyć wszelkich starań w celu utrzymania zapasu środków na~kontach Stowarzyszenia wystarczającego na pokrycie stałych zobowiązań Stowarzyszenia przez okres co najmniej trzech miesięcy.
		\item Do zawierania umów, udzielania pełnomocnictwa i składania innych oświadczeń woli, w szczególności w sprawach majątkowych:
		\begin{enumerate}
			\item upoważnionych jest dowolnych dwóch członków Zarządu działających łącznie,
			\item upoważniony jest każdy członek Zarządu działający samodzielnie, jeżeli wysokość podejmowanego zobowiązania nie przekracza 128 PLN.
		\end{enumerate}
	\end{enumerate}
	

\suspend{enumerate}
\section{Postanowienia końcowe}
\resume{enumerate}
	
	\item \begin{enumerate}
		\item Uchwałę w sprawie zmiany Statutu oraz uchwałę o rozwiązaniu Stowarzyszenia podejmuje Walne Zebranie Członków kwalifikowaną większością głosów (2/3) przy obecności co najmniej połowy łącznej liczby członków Stowarzyszenia uprawnionych do głosowania, stanowiących kworum.
		\item Podejmując uchwałę o rozwiązaniu Stowarzyszenia, Walne Zebranie Członków określa sposób jego likwidacji oraz przeznaczenie majątku Stowarzyszenia.
	\end{enumerate}		
	
	\item  W sprawach nieuregulowanych w niniejszym statucie zastosowanie mają przepisy ustawy Prawo o~stowarzyszeniach.
			
\end{enumerate}


\end{document}
